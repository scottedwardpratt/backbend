\documentclass[12pt]{article}
%#Scott Pratt, Eren Erdogan, Ekaksh Kataria 2023
\usepackage{subfiles}
%\usepackage{indentfirst}
\usepackage{graphicx}
\usepackage[
        pdfencoding=auto,%
        pdftitle={Smooth Emulator and Simplex Sampler}
        pdfauthor={Scott Pratt},%
        pdfstartview=FitV,%
        colorlinks=true,%
        linkcolor=blue,%
        citecolor=blue, %
        urlcolor=blue,
				breaklinks=true]{hyperref}
%\usepackage[anythingbreaks,hyphenbreaks]{breakurl}
\usepackage{xurl}
\usepackage{comment}
%\usepackage{pdfsync}
\usepackage{amssymb}
\usepackage{amsmath}
%\usepackage{nopageno}
\usepackage{bm}
\usepackage{dsfont}
%\usepackage[utf8]{inputenc}
\usepackage[small,bf]{caption}
%\usepackage{fontspec}
%\usepackage{textcomp}
%\usepackage{color}
%\usepackage{fancyhdr}
\usepackage[titletoc]{appendix}
%\usepackage[headheight=110pt]{geometry}
\usepackage{bm}

\numberwithin{equation}{section} 
\numberwithin{figure}{section} 

%\usepackage[most]{tcolorbox}
%\tcbset{
%frame code={}
%center title,
%left=0pt,
%right=0pt,
%top=0pt,
%bottom=0pt,
%colback=gray!25,
%colframe=white,
%width=\dimexpr\textwidth\relax,
%enlarge left by=0mm,
%boxsep=5pt,
%arc=0pt,outer arc=0pt,
%}
%\newcounter{examplecounter}
%\counterwithin{examplecounter}{section}
%\setcounter{examplecounter}{0}
%\newcommand{\example}[2]{\begin{tcolorbox}[breakable,enhanced]
%\refstepcounter{examplecounter}{
%\bf Example \arabic{section}.\arabic{examplecounter}:}~~{\bf #1}\\
%{#2}
%\end{tcolorbox}
%}

%\newcommand{\exampleend}{
%\begin{samepage}
%\nopagebreak\noindent\rule{\textwidth}{1pt}
%\end{samepage}
%}

%\usepackage{silence}
%\WarningFilter{hyperref}{Token not allowed in a PDF String}

\newcommand\eqnumber{\addtocounter{equation}{1}\tag{\theequation}}
%\newcommand{\solution}[1]{ }
\newcommand\identity{\mathds{1}}

\setlength{\headheight}{16pt}
\parskip 6pt
\parindent 0pt
\textwidth 7.0in
\hoffset -0.8in
\textheight 9.2in
\voffset -1in

%\newcommand{\bm}{\boldmath}
\boldmath
%
\begin{document}
The CC equation is
\begin{align*}
\frac{dP}{dT}&=\frac{\sigma_{\rm gas}-\sigma_{\rm liq}}{\rho^{-1}_{\rm gas}-\rho^{-1}_{\rm liq}}\equiv \alpha.
\end{align*}
The quantity $\alpha$ can be thought of as the latent heat (per charge) between the two phases divided by volume (per charge) of the two phases.

\begin{align*}
\delta P_{\rm gas}&=\frac{\partial P_{\rm gas}}{\partial\rho_{\rm gas}}\delta \rho_{\rm gas}+\frac{\partial P_{\rm gas}}{\partial T_{\rm gas}}\delta T_{\rm gas},\\
\alpha\delta T_{\rm gas}&=\frac{\partial P_{\rm gas}}{\partial\rho_{\rm gas}}\delta \rho_{\rm gas}+\frac{\partial P_{\rm gas}}{\partial T_{\rm gas}}\delta T_{\rm gas},\\
\frac{\partial P_{\rm gas}}{\partial\rho_{\rm gas}}\delta\rho_{\rm gas}&=\left[\alpha-\frac{\partial P_{\rm gas}}{\partial T_{\rm gas}}\right]\delta T_{\rm gas},\\
\frac{d\rho_{\rm gas}}{dT}&=\frac{1}{\partial P_{\rm gas}/{\partial\rho_{\rm gas}}}
\left[\alpha-\frac{\partial P_{\rm gas}}{\partial T_{\rm gas}}\right]\\
&=\frac{\partial \rho_{\rm gas}/\partial\mu_{\rm gas}}{\partial P_{\rm gas}/{\partial\mu_{\rm gas}}}
\left[\alpha-\frac{\partial P_{\rm gas}}{\partial T_{\rm gas}}\right]\\
&=\frac{\mathcal{F}_{\rm gas}}{T}\left[\alpha-\frac{\partial P_{\rm gas}}{\partial T_{\rm gas}}\right],\\
\mathcal{F}_{\rm gas}&=\frac{\langle(\delta Q)^2\rangle_{\rm gas}}{\langle Q\rangle_{\rm gas}}.
\end{align*}
Here $\mathcal{F}_{\rm gas}$ is the charge fluctuation, and is unity for a Poissonian charge limit. Near the critical point it approaches infinity. It is always positive. 
Next, derive some thermodynamic relations for $\partial P/\partial T|_\rho$,
\begin{align*}
\delta P&=s\delta T +\rho\delta\mu,\\
\delta\rho&=\left.\frac{\partial\rho}{\partial T}\right|_\mu\delta T+\frac{\rho}{T}\mathcal{F}\delta \mu=0,\\
\delta P|_\rho&=\delta T\left[s-T\frac{\partial\rho/\partial T}{\mathcal{F}}\right],\\
\left.\frac{\partial P}{\partial T}\right|_\rho&=s-T\frac{\partial\rho/\partial T|_\mu}{\mathcal{F}}
\end{align*}


where $s$ is the entropy density, this gives
\begin{align*}
\frac{d\rho_{\rm gas}}{dT}&=\frac{\mathcal{F}_{\rm gas}}{T}\left[\alpha-s_{\rm gas}\right]+\left.\frac{\partial\rho_{\rm gas}}{\partial T}\right|_\mu\\
&=\frac{\mathcal{F}_{\rm gas}}{T}\left[\alpha-s_{\rm gas}\right]
+\left.\frac{\partial\rho_{\rm gas}}{\partial T}\right|_{\mu/T}
-\frac{\mu}{T^2}\frac{\partial\rho_{\rm gas}}{\partial (\mu/T)}\\
&=\frac{\mathcal{F}_{\rm gas}}{T}\left[\alpha-\frac{P_{\rm gas}+\epsilon_{\rm gas}}{T}\right]
+\left.\frac{\partial\rho_{\rm gas}}{\partial T}\right|_{\mu/T}.
\end{align*}
The last term can also be equated to the charge-energy fluctuation,
\begin{align*}
\left.\frac{\partial\rho_{\rm gas}}{\partial T}\right|_{\mu/T}&=\frac{1}{VT^2}\langle\delta E\delta Q\rangle_{\rm gas}
\end{align*}

\begin{comment}
Some rearranging,
\begin{align*}
\alpha-s_{\rm gas}&=\frac{s_{\rm gas}\rho_{\rm lig}-s_{\rm lig}\rho_{\rm gas}}{\rho_{\rm liq}-\rho_{\rm gas}}-s_{\rm gas}\\
&=\frac{s_{\rm gas}\rho_{\rm lig}-s_{\rm lig}\rho_{\rm liq}}{\rho_{\rm liq}-\rho_{\rm gas}}
-\frac{s_{\rm gas}(\rho_{\rm liq}-\rho_{\rm gas})}{\rho_{\rm liq}-\rho_{\rm gas}}\\
&=\frac{s_{\rm gas}\rho_{\rm gas}-s_{\rm liq}\rho_{\rm gas}}{\rho_{\rm liq}-\rho_{\rm gas}}\\
&=\frac{s_{\rm gas}-s_{\rm liq}}{\rho_{\rm liq}-\rho_{\rm gas}}\rho_{\rm gas}.
\end{align*}
The result is:
\begin{align*}
\frac{d\rho_{\rm gas}}{dT}&=\frac{\mathcal{F}_{\rm gas}}{T}\frac{(s_{\rm gas}-s_{\rm liq})}{(\rho_{\rm liq}-\rho_{\rm gas})}\rho_{\rm gas} +\left.\frac{\partial\rho_{\rm gas}}{\partial T}\right|_\mu.
\end{align*}
\end{comment}

\end{document}

